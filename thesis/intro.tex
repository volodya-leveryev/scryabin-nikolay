\section*{Введение}
\addcontentsline{toc}{section}{Введение}

\textbf{Актуальность}. Опишите текущее состояние проблемы, которую вы собираетесь решить в работе. Примерный объем: 1-3 абзаца по 2-4 предложения в каждом.

\textbf{Цель работы} — 1 предложение которое должно отражать тему работы.

\textbf{Задачи работы} — от 2 до 4 задач, которые соответствуют этапам достижения цели работы. Каждая задача должна стать главой или значимым разделом в тексте работы.

\textbf{Объект и предмет исследования}. Объект исследования это нечто (объект, процесс или явление), что порождает проблему. Объект исследования отвечает на вопрос "Что исследуется?". Объект исследования, в отличии от предмета исследования, является более общим понятием. Предмет исследования это конкретный аспект проблемы или частное свойство объекта исследования. Предмет исследования зачастую совпадает с темой работы.

\textbf{Краткое содержание работы}. Одно предложение — перечисление разделов из которых состоит данная работа. Далее — краткое описание содержания каждой главы работы без учёта введения, заключения и списка использованных источников.
